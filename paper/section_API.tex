\section{API}
W tej sekcji szczegółowo opisane zostały udostępnione przez \texttt{dhttp.js} funkcje i mechanizmy związane ze strukturami i przepływami danych pomiędzy węzłami. W przypadku wątpliwości co do działania pewnych rozwiązań czy logicznych połączeń między nimi, warto odnieść się do podsekcji \ref{sub:flows}, opisującej przepływy danych, oraz podsekcji \ref{sub:nodeCycle} opisującej cykl życia węzła.

Podstawowym założeniem jest istnienie singletona \texttt{dhttpClient}, który przy rozruchu aplikacji inicjalizuje węzeł sieci. Wszelkie odwołania do działania węzła powinny polegać na rejestrowaniu wydarzeń i wywoływaniu funkcji globalnego obiektu \texttt{dhttpClient}.

\subsection{Struktury danych}
W celu poprawy wydajności i uproszczenia przepływów w aplikacji, dHTTP oferuje kilka struktur informujących o dostępności danych.

Typy struktur jak i funkcje oferowane przez \texttt{js-libp2p}, do których odnosi się poniższe API, opisane są na stronie projektu (\cite{libp2pReadme}).

\subsubsection{Struktury podstawowe}
\begin{itemize}
    \item \textbf{\texttt{PeerId}} -- \texttt{String} zawierający ID węzła sieci, uzyskane dzięki zawołaniu \texttt{PeerInfo.id.toB58String()}
    \item \textbf{\texttt{RequestURL}} -- \texttt{String} zawierający ścieżkę zapytania klienckiego. Jeśli klient odpyta nasz serwer na adresie \texttt{http://localhost:34887/http://example.com/script.js}, instancja \texttt{RequestURL} zawierać będzie \texttt{http://example.com/script.js}.
\end{itemize}

% 

\subsubsection{\texttt{filesInNetwork}}
\begin{lstlisting}[language=javascript]
filesInNetwork: {
        "http://example.com/file.jpg": { // RequestURL
            nodes: new Set([PeerId]),
            popularityIndex: 0
        },
        "http://example.com/script.js" : {...}
    }
\end{lstlisting}
\texttt{filesInNetwork} jest strukturą odpytywaną przy połączeniu klienckim. Stanowi pierwszy punkt wejścia do sieci, i zoptymalizowany jest pod szybkie przeszukanie i weryfikację, czy plik jest dostępny przy użyciu dHTTP.

Poszczególne elementy indeksowane są poprzez adresy URL zasobów pobieranych przez klienta (\texttt{ReqestURL}).

\textbf{Pola:}
\begin{itemize}
    \item \textbf{\texttt{nodes}} -- \texttt{Set} zawierający \texttt{PeerId} tych klientów, którzy posiadają dostęp do danego pliku.
    \item \textbf{\texttt{popularityIndex}} -- liczba całkowita określająca popularność pliku z perspektywy tego węzła. Może być wymieniana z innymi węzłami sieci w celu optymalizacji obciążenia.
\end{itemize}

% 

\subsubsection{\texttt{swarm}}
\begin{lstlisting}[language=javascript]
swarm: { 
    nodes: { 
        QmVxxoRFLR8VjbqeVer9Z9DqCJoJT36c9Uomd9AtP8NBx6: // PeerId
            PeerInfo {...}
    } 
}
\end{lstlisting}
\texttt{swarm} jest strukturą utrzymującą informacje na temat wszystkich węzłów, które są bezpośrednio podłączone do obecnego. Reprezentuje więc swoisty {\em lokalny klaster}, z całością którego można nawiązywać połączenia.

\textbf{Pola:}
\begin{itemize}
    \item \textbf{\texttt{nodes}} -- obiekt, w którym poszczególne elementy reprezentują \texttt{PeerInfo} -- kompleksowy zbiór informacji z API \texttt{js-libp2p}, pozwalający na nawiązywanie połączeń.
\end{itemize}

% 

\subsubsection{\texttt{stats}}
\begin{lstlisting}[language=javascript]
stats: {
    downloaded: {bytes: 0, files: 0},
    fetched: {bytes: 0, files: 0},
    uploaded: {bytes: 0, files: 0},
}
\end{lstlisting}
\texttt{stats} gromadzi statystyki zużycia sieci przez obecny węzeł.

\textbf{Pola:}
\begin{itemize}
    \item \textbf{\texttt{downloaded}} -- ile danych pobrano z perspektywy klienta.
    \item \textbf{\texttt{fetched}} -- ile danych pobrano w tle -- podczas operacji, niezwiązanych bezpośrednio z działania użytkownika, lecz podjętych w celach optymalizacyjnych.
    \item \textbf{\texttt{uploaded}} -- ile danych wysłano do innych węzłów podczas działań optymalizacyjnych i propagacji danych w systemie.
\end{itemize}

% 

\subsubsection{\texttt{node}}
\begin{lstlisting}[language=javascript]
    node: Node {protocols: {...}, peerInfo: PeerInfo, ...}
\end{lstlisting}
\texttt{node} jest obiektem pozwalającym na bezpośredni dostęp do reprezentacji danego węzła w warstwie \texttt{js-libp2p}. Pozwala na wykonywanie połączeń i propagację danych z poziomu innych funkcji dHTTP.

% 

\subsubsection{\texttt{storage}}
\begin{lstlisting}[language=javascript]
storage: {
    quotas: {
        RAM: {
            files: 0, megabytes: 0
        },
        drive: {
            files: 0, megabytes: 0
        }
    },
    RAMFiles: {
        files: {
            "http://example.com/logo.png": { // RequestURL
                url: "http://example.com/logo.png", // RequestURL
                ETag: "w/kjadDJKj",
                value: Blob
            },
            "http://example.com/logoLocalStorage.png": { ... },
            ...
        }
    },
    driveFiles: {
        files: {
            "http://example.com/onDrive.png": { // RequestURL
                url: "http://example.com/onDrive.png", // RequestURL
                ETag: "w/kjadDJKj"
            },
            "http://example.com/onDrive2.png": { ... },
            ...
        }
    }
}
\end{lstlisting}
\texttt{storage} nakłada warstwę abstrakcji na obsługę plików utrzymywanych przez węzeł, która automatyzuje proces wyboru rodzaju pamięci, czy odczytów z różnych źródeł.

\textbf{Pola:}
\begin{itemize}
    \item \textbf{\texttt{quotas}} -- obiekt który zawiera w sobie konfigurację ograniczeń pamięci. Dla wartości \texttt{0} nie istnieją żadne ograniczenia; wartość \texttt{-1} oznacza zakaz korzystania z danego rodzaju pamięci.
    \item \textbf{\texttt{RAMFiles}} -- zbiór wszystkich plików przechowywanych w pamięci 
\end{itemize}


\subsection{Funkcje}
\label{sub:functions}

\subsection{Wydarzenia}
% to może nie być aż tak strict; może lub musi jednak odnosić się do istniejących funkcji
\subsection{Komunikaty metaprotokołu}
% TODO -- umieść wszystkie możliwe komunikaty metaprotokołu na liście
\subsection{Połączenia}