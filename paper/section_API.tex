\section{API}
\label{sec:api}

W tej sekcji szczegółowo opisane zostały udostępnione przez \texttt{dhttp.js} funkcje i mechanizmy związane ze strukturami i przepływami danych pomiędzy węzłami. Ogólny przepływ informacji systemu zawiera się w do podsekcji \ref{sub:flows}, opisującej przepływy danych, oraz podsekcji \ref{sub:nodeCycle}, opisującej cykl życia węzła.

Podstawowym założeniem jest istnienie singletona \texttt{dhttpClient}, który przy rozruchu aplikacji inicjalizuje węzeł sieci. Wszelkie odwołania do działania węzła powinny polegać na rejestrowaniu wydarzeń i wywoływaniu funkcji globalnego obiektu \texttt{dhttpClient}.

\subsection{Struktury danych}
W celu poprawy wydajności i uproszczenia przepływów w aplikacji, dHTTP oferuje kilka struktur informujących o dostępności danych.

Typy struktur jak i funkcje oferowane przez \texttt{js-libp2p}, do których odnosi się poniższe API, opisane są na stronie projektu: \cite{libp2pReadme}.

\subsubsection{Struktury podstawowe}
\begin{itemize}
    \item \textbf{\texttt{PeerId}} -- \texttt{String} zawierający ID węzła sieci, uzyskane dzięki zawołaniu \texttt{PeerInfo.id.toB58String()}
    \item \textbf{\texttt{RequestURL}} -- \texttt{String} zawierający ścieżkę zapytania klienckiego. Jeśli klient odpyta nasz serwer na adresie \texttt{http://localhost:34887/http://example.com/script.js}, instancja \texttt{RequestURL} zawierać będzie \texttt{http://example.com/script.js}.
    \item \textbf{\texttt{popularityIndex}} -- \texttt{Integer} określający popularność danego pliku z perspektywy posiadającego go węzła.
    \item \textbf{\texttt{Data}} -- serializowalny obiekt zawierający dane przesyłane w ramach metaprotokołu, zgodne z formatami zdefiniowanymi w \ref{sub:metaprotocolMessages}.
\end{itemize}

% 

\subsubsection{\texttt{filesInNetwork}}
\begin{lstlisting}[language=javascript]
filesInNetwork: {
        "http://example.com/file.jpg": { // RequestURL
            nodes: {
                PeerId: {
                    popularityIndex: Int,
                }
            }
            popularityIndex: Int
        },
        "http://example.com/script.js" : {...}
    }
\end{lstlisting}
\texttt{filesInNetwork} jest strukturą odpytywaną przy połączeniu klienckim. Stanowi pierwszy punkt wejścia do sieci, i zoptymalizowany jest pod szybkie przeszukanie i weryfikację, czy plik jest dostępny przy użyciu dHTTP.

Poszczególne elementy indeksowane są poprzez adresy URL zasobów pobieranych przez klienta (\texttt{ReqestURL}).

\textbf{Pola:}
\begin{itemize}
    \item \textbf{\texttt{nodes}} -- mapa zawierająca informacje tych klientów, którzy posiadają dostęp do danego pliku.
    \item \textbf{\texttt{popularityIndex}} -- liczba całkowita określająca popularność pliku z perspektywy tego węzła. Może być wymieniana z innymi węzłami sieci w celu optymalizacji obciążenia -- ułatwi węzłowi odpytywanemu decyzję na temat przekazania połączenia do najmniej zajętego sąsiada.
\end{itemize}

% 

\subsubsection{\texttt{swarm}}
\begin{lstlisting}[language=javascript]
swarm: { 
    nodes: { 
        QmVxxoRFLR8VjbqeVer9Z9DqCJoJT36c9Uomd9AtP8NBx6: // PeerId
            PeerInfo {...}
    } 
}
\end{lstlisting}
\texttt{swarm} jest strukturą utrzymującą informacje na temat wszystkich węzłów, które są bezpośrednio podłączone do obecnego. Reprezentuje więc swoisty {\em lokalny klaster}, z całością którego można nawiązywać połączenia.

\textbf{Pola:}
\begin{itemize}
    \item \textbf{\texttt{nodes}} -- obiekt, w którym poszczególne elementy reprezentują \texttt{PeerInfo} -- kompleksowy zbiór informacji z API \texttt{js-libp2p}, pozwalający na nawiązywanie połączeń.
\end{itemize}

% 

\subsubsection{\texttt{stats}}
\begin{lstlisting}[language=javascript]
stats: {
    downloaded: {bytes: 0, files: 0},
    fetched: {bytes: 0, files: 0},
    uploaded: {bytes: 0, files: 0},
}
\end{lstlisting}
\texttt{stats} gromadzi statystyki zużycia sieci przez obecny węzeł.

\textbf{Pola:}
\begin{itemize}
    \item \textbf{\texttt{downloaded}} -- ile danych pobrano z perspektywy klienta.
    \item \textbf{\texttt{fetched}} -- ile danych pobrano w tle -- podczas operacji niezwiązanych bezpośrednio z działania użytkownika, lecz podjętych w celach optymalizacyjnych.
    \item \textbf{\texttt{uploaded}} -- ile danych wysłano do innych węzłów podczas działań optymalizacyjnych i propagacji danych w systemie.
\end{itemize}

% 

\subsubsection{\texttt{node}}
\begin{lstlisting}[language=javascript]
    node: Node {protocols: {...}, peerInfo: PeerInfo, ...}
\end{lstlisting}
\texttt{node} jest obiektem pozwalającym na bezpośredni dostęp do reprezentacji danego węzła w warstwie \texttt{js-libp2p}. Pozwala na wykonywanie połączeń i propagację danych z poziomu innych funkcji dHTTP. Szczegóły zawarte w dokumentacji projektu (\cite{libp2pReadme}).

% 

\subsubsection{\texttt{storage}}
\begin{lstlisting}[language=javascript]
storage: {
    quotas: {
        RAM: {
            files: 0, megabytes: 0
        },
        drive: {
            files: 0, megabytes: 0
        }
    },
    RAMFiles: {
        files: {
            "http://example.com/logo.png": { // RequestURL
                value: Blob
            },
            "http://example.com/logoLocalStorage.png": { ... },
            ...
        }
    },
    driveFiles: {
        files: {
            "http://example.com/onDrive.png": { // RequestURL
                path: "/files/http://example.com/onDrive.png"
            },
            "http://example.com/onDrive2.png": { ... },
            ...
        }
    }
}
\end{lstlisting}
\texttt{storage} nakłada warstwę abstrakcji na obsługę plików utrzymywanych przez węzeł, która automatyzuje proces wyboru rodzaju pamięci dla zapisu oraz odczytu danych.

\textbf{Pola:}
\begin{itemize}
    \item \textbf{\texttt{quotas}} -- obiekt który zawiera w sobie konfigurację ograniczeń pamięci. Dla wartości \texttt{0} nie istnieją żadne ograniczenia; wartość \texttt{-1} oznacza zakaz korzystania z danego rodzaju pamięci.
    \item \textbf{\texttt{RAMFiles}} -- zbiór wszystkich plików przechowywanych w pamięci RAM.
    \subitem  \textbf{\texttt{value}} -- plik trzymany w formacie binarnym, gotowy do strumieniowego przesłania użytkownikowi końcowemu.
    \item \textbf{\texttt{RAMFiles}} -- zbiór wszystkich plików przechowywanych w systemie plików użytkownika.
    \subitem  \textbf{\texttt{value}} -- względna ścieżka do pliku, który zawiera dany zasób.
\end{itemize}


%  FUNKCJE

\subsection{Funkcje}
\label{sub:functions}

Dzięki szeregowi udostępnionych funkcji, dHTTP umożliwia wysokopoziomowe, asynchroniczne podejście do informacji o systemie i zawartości plików. Poniżej zostały zawarte opisy najważniejszych z nich.

\subsubsection{\texttt{filesAvailable/0}}
\begin{lstlisting}[language=javascript]
    filesAvailable: function() -> ({RequestURL: {PopularityIndex}})
\end{lstlisting}
\texttt{filesAvailable} jest funkcją stanowiącą wykaz wszystkich plików udostępnionych przez dany węzeł. Nie stosuje rozróżnienia na dane dostępne w RAMie czy na dysku -- ta optymalizacja pozostaje na poziomie poszczególnych węzłów, aby nie obciążać wszystkich klientów sieci.

\textbf{Wartość zwracana:}
Mapa \texttt{RequestURL} obsługiwanych przez dany węzeł na ich \texttt{PopularityIndex}.

% \textbf{Przykład wywołania:}
% \begin{lstlisting}[language=javascript]
%     introductoryData: function () {
%     return {
%         type: "introduction",
%         peerId: dhttpClient.node.peerInfo.id.toB58String(),
%         files: dhttpClient.filesAvailable()
%     }
% }
% \end{lstlisting}


% 

\subsubsection{\texttt{dialNode/3}}
\begin{lstlisting}[language=javascript]
    dialNode: function(PeerInfo, Data, (ResponseData) -> ()) -> ()
\end{lstlisting}
\texttt{dialNode} pozwala na wykonanie połączenia do istniejącego węzła w celu przesłania mu metadanych projektu. W ramach danych, wysłane mogą zostać serializowane obiekty komunikatów, opisane szczegółowo w sekcji \ref{sub:metaprotocolMessages}.

\textbf{Parametry:}
\begin{itemize}
    \item \textbf{\texttt{PeerInfo}} -- obiekt \texttt{PeerInfo} węzła do którego ma zostać wysłana wiadomość
    \item \textbf{\texttt{Data}} -- serializowalny obiekt zawierający wysyłane dane; \texttt{ResponseData} jest tym samym typem, wydzielonym w celu większej czytelności -- zaznaczenia, że reprezentuje dane otrzymane.
    \item \textbf{\texttt{(ResponseData) -> ()}} -- callback, wołany dla odpowiedzi otrzymywanych od serwera. Otrzymuje jako parametr zdeserializowany obiekt odpowiedzi.
\end{itemize}

% 

\subsubsection{\texttt{addFile/3}}
\begin{lstlisting}[language=javascript]
    addFile: function(RequestURL, popularityIndex, (ResponseData) -> ()) -> ()
\end{lstlisting}
\texttt{addFile} pozwala poinformować sieć o pojawieniu się nowego pliku i o tym, że jest on obsługiwany przez nas, z danym indeksem popularności. Sieć może wykorzystać tę wiedzę aby zacząć przekierowywać użytkowników na dany plik, lub pobierać go w celach udostępniania.

\textbf{Parametry:}
\begin{itemize}
    \item \textbf{\texttt{RequestURL}} -- ścieżka pliku
    \item \textbf{\texttt{popularityIndex}} -- patrz: \texttt{filesInNetwork}
    \item \textbf{\texttt{(ResponseData) -> ()}} -- callback, wołany dla odpowiedzi otrzymywanych od serwera; pozwala reagować w przypadkach wystąpienia nieoczekiwanych błędów.
\end{itemize}

% 

\subsubsection{\texttt{removeFile/2}}
\begin{lstlisting}[language=javascript]
    removeFile: function(RequestURL, (ResponseData) -> ()) -> ()
\end{lstlisting}
\texttt{removeFile} pozwala poinformować sieć o decyzji zakończenia udostępnienia danego pliku. Może mieć związek z decyzjami optymalizacyjnymi -- jeżeli danemu węzłowi zaczyna brakować miejsca, może podjąć decyzję o usuwaniu najmniej popularnych plików.

\textbf{Parametry:}
\begin{itemize}
    \item \textbf{\texttt{RequestURL}} -- ścieżka pliku
    \item \textbf{\texttt{(ResponseData) -> ()}} -- callback, wołany dla odpowiedzi otrzymywanych od serwera; pozwala reagować w przypadkach wystąpienia nieoczekiwanych błędów.
\end{itemize}

% 

\subsubsection{\texttt{fetchFromNode/3}}
\begin{lstlisting}[language=javascript]
    fetchFromNode: function(RequestURL, Writable, PeerInfo) -> ()
\end{lstlisting}
\texttt{fetchFromNode} pozwala pobierać pliki. Jest dedykowane pobraniom wewnętrznym, optymalizacyjnym -- powinno służyć węzłowi do aktualizacji własnej bazy plików. Funkcja wysyła \texttt{end} do strumienia pytającego w momencie zakończenia przesyłu.

\textbf{Parametry:}
\begin{itemize}
    \item \textbf{\texttt{RequestURL}} -- ścieżka pliku
    \item \textbf{\texttt{Writable}} -- strumień, do którego ma zostać wysłany poszukiwany plik.
    \item \textbf{\texttt{PeerInfo}} -- obiekt reprezentujący węzeł, z którego pobieramy plik.
\end{itemize}


% 

\subsubsection{\texttt{propagate/2}}
\begin{lstlisting}[language=javascript]
    propagate: function(Data, (Data) -> ()) -> ()
\end{lstlisting}
\texttt{propagate} wysyła dane przy użyciu metaprotokołu do wszystkich dostępnych węzłów. \ref{{sub:metaprotocolMessages}} zawiera szczegóły obsługiwanych komunikatów.

\textbf{Parametry:}
\begin{itemize}
    \item \textbf{\texttt{Data}} -- przesyłane, serializowalne dane.
    \item \textbf{\texttt{(ResponseData) -> ()}} -- callback, wołany dla odpowiedzi otrzymywanych od serwera; pozwala reagować w przypadkach wystąpienia nieoczekiwanych błędów.
\end{itemize}

% 

\subsection{Wydarzenia}
\label{sec:events}
Ta sekcja zawiera funkcje o wspólnej charakterystyce -- są częścią {\em reaktywną} systemu, wołane są poprzez odwołania w czasie połączeń między węzłami i stanowią definicję reakcji na sytuacje i zapytania występujące w systemie.

% 

\subsubsection{\texttt{download/3}}
\begin{lstlisting}[language=javascript]
     download: function(RequestURL, Writable, Fallback () -> ()) -> ()
\end{lstlisting}
\texttt{download} jest funkcją pozwalającą na pobranie pliku z sieci, z uwzględnieniem źródeł danych i sąsiadów węzła. Szczegóły cyklu pobierania danych omówione są w sekcji \ref{sec:transportAlgorithm}.

\textbf{Parametry:}
\begin{itemize}
    \item \textbf{\texttt{RequestURL}} -- adres żądanego pliku.
    \item \textbf{\texttt{Writable}} -- strumień do którego powinien zostać zapisany plik. Otrzymuje \texttt{end} w momencie zakończenia pliku.
    \item \textbf{\texttt{Fallback}} -- funkcja która może zostać zawołana w przypadku nieznalezienia pliku w zasięgu węzła -- przykładowo, możliwe jest dzięki niej wysłanie pliku pobranego bezpośrednio z oryginalnego serwera.
\end{itemize}

% 

\subsubsection{\texttt{acceptIntroduction/2}}
\begin{lstlisting}[language=javascript]
    acceptIntroduction: function(ResponseData) -> ()
\end{lstlisting}
\texttt{acceptIntroduction} służy obsłudze sytuacji, gdy otrzymamy informację o pojawieniu się węzła sieci -- propaguje on \texttt{cluster} i \texttt{filesInNetwork} szczegółami na temat nowego sąsiada i oferowanych przez niego plików. 

\textbf{Parametry:}
\begin{itemize}
    \item \textbf{\texttt{ResponseData}} -- przesyłane przez węzeł, serializowalne metadane typu \texttt{introduction}.
\end{itemize}

% 

\subsubsection{\texttt{attachNode/2}}
\begin{lstlisting}[language=javascript]
    attachNode: function (PeerInfo) -> ()
\end{lstlisting}
\texttt{attachNode} służy dodaniu połączenia do węzła, i wywołaniu \texttt{introduceSelf/1} -- odesłaniu nowemu sąsiadowi szczegółów na temat samego siebie w postaci metadanych typu \texttt{introduction}.

\textbf{Parametry:}
\begin{itemize}
    \item \textbf{\texttt{PeerInfo}} -- obiekt pozwalający na nawiązywanie połączeń.
\end{itemize}

% 

\subsubsection{\texttt{detachNode/2}}
\begin{lstlisting}[language=javascript]
    detachNode: function (PeerInfo) -> ()
\end{lstlisting}
\texttt{detachNode} pozwala usunąć obiekt z mapy \texttt{swarm} oraz informacje o istniejących plikach, efektywnie wypinając dany węzeł z własnej sieci.

\textbf{Parametry:}
\begin{itemize}
    \item \textbf{\texttt{PeerInfo}} -- obiekt reprezentujący usuwany węzeł.
\end{itemize}

% 




% middle interfaces and stuff: attachNode, detachNode, storage.storeFile, download
% to może nie być aż tak strict; może lub musi jednak odnosić się do istniejących funkcji
\subsection{Komunikaty metaprotokołu}
\label{sub:metaprotocolMessages}

Metaprotokół dHTTP służy prostej propagacji danych na temat węzłów i plików dostępnych w systemie. Sekcja \ref{sub:metadata} wprowadza podstawowe zapytanie dHTTP; poniżej omówienie stosowanych w systemie komunikatów i ich reakcji.

% 

\subsubsection{\texttt{echo}}
\begin{lstlisting}[language=javascript]
    {
        type: "echo",
        % peerId: dhttpClient.node.peerInfo.id.toB58String(), // ID pytającego
        text: "Hello, node."
    }
\end{lstlisting}

Komunikat stosowany w celach debugowania i sprawdzenia połączenia między węzłami, odpowiednik polecenia \texttt{ping}. Odpowiedź powinna zawierać analogiczną zawartość.

% 

\subsubsection{\texttt{introduction}}
\begin{lstlisting}[language=javascript]
    {
        type: "introduction",
        peerId: dhttpClient.node.peerInfo.id.toB58String(),
        files: dhttpClient.filesAvailable()
    }
\end{lstlisting}

Komunikat informujący o plikach dostępnych w danym węźle. Pozwala na uzupełnienie informacji na temat dostępności plików. Ponieważ wszystkie węzły rozsiewają \texttt{introduction} sobie nawzajem, odpowiedź nie jest wymagana.

% 

\subsubsection{\texttt{hasFile}}
\begin{lstlisting}[language=javascript]
    {
        type: "hasFile",
        peerId: dhttpClient.node.peerInfo.id.toB58String(),
        files: RequestURL
    }
\end{lstlisting}

Komunikat pytający, czy dany węzeł posiada plik lub zna węzeł który może ten plik posiadać.

\textbf{Oczekiwana odpowiedź:}
\begin{lstlisting}[language=javascript]
    {
        type: "hasFile",
        peerId: dhttpClient.node.peerInfo.id.toB58String(),
        hasFile: true/false,
        popularityIndex: PopularityIndex,
        peerWithFile: PeerInfo // jeśli plik nie jest posiadany przez dany węzeł, ale wie on o potencjalnym posiadaczu
    }
\end{lstlisting}

% 

\subsubsection{\texttt{addFile}}
\begin{lstlisting}[language=javascript]
    {
        type: "addFile",
        peerId: dhttpClient.node.peerInfo.id.toB58String(),
        files: RequestURL,
        popularityIndex: PopularityIndex
    }
\end{lstlisting}

Komunikat informujący o pojawieniu się nowego pliku; może być używany wielokrotnie dla tego samego pliku w celu aktualizacji informacji o \texttt{PopularityIndex}. Jest propagowany w całej sieci; jeśli inny węzeł posiada ten sam plik, powinien odpowiedzieć analogicznie.


% 

\subsubsection{\texttt{removeFile}}
\begin{lstlisting}[language=javascript]
    {
        type: "removeFile",
        peerId: dhttpClient.node.peerInfo.id.toB58String(),
        files: RequestURL
    }
\end{lstlisting}

Komunikat informujący o usunięciu pliku z danego węzła, propagowany w całe sieci w celu aktualizacji tablicy plików.


% TODO - rysunek przedstawiający całe API -- możliwe, że posłuży za niego opis w kolejnej sekcji.