\chapter{Projektowanie i implementacja}
\label{cha:implementacja}



\section{Wymagania funkcjonalne}
\label{sec:funkcjonalnosc}

\section{Definicje, architektura i technologie}
\label{sec:zalozeniaProjektu}

\subsection{Wykorzystane technologie i narzędzia}
\label{sec:techNTools}

\subsection{Problemy warstwy sieciowej}
\label{sec:networkIssues}

\subsection{Decentralizacja czy rozproszenie?}
\label{sec:decentralizacjaCzyRozproszenie}

\subsection{Bezpieczeństwo}
\label{sec:security}

\subsubsection{Czy autentykacja to nasza brożka?}
\subsubsection{Gdzie leżą granice zdrowych heurystyk?}

\subsection{Propagacja i przechowywanie danych}
\label{sec:dataPropagation}

% Z tego poniżej może powstać osobny rozdział, gdzie to co powyżej stanowiłoby bardziej rozważania na temat właściwego podejścia do realizacji projektu, a to poniżej -- faktyczny opis implementacji systemu, podzielony bardziej granularnie dla każdego z elemenetów

\section{Aplikacje}

\subsection{Węzeł}
\subsection{Klient}

