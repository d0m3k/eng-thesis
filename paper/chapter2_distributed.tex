\chapter{Droga do rozproszenia}
\label{cha:rozproszenie}

Osławione prawo Moore'a (\cite{moore1998cramming}), wspominające o podwajaniu ilości tranzystorów w procesorach, często parafrazowane jako podwajanie ich mocy obliczeniowej, przestaje działać, podczas gdy ilość użytkowników szerokopasmowego internetu rośnie. 
Próba skalowania wertykalnego -- polegającego na wzmacnianiu pojedynczych węzłów, przestaje zdawać próbę czasu.

Problem posiada także drugą stronę -- nie wszyscy użytkownicy internetu mogą pozwolić sobie na łącze szerokopasmowe. Szacuje się, że w roku 2017 dostęp do internetu posiada ponad połowa populacji świata; sam wzrost ilości użytkowników sieci Web z Afryki od roku 2000 do 2017 wynosi aż 8503.1\% (\cite{webStats}) – znaczną część tych połączeń stanowią jednak połączenia starych generacji sieci komórkowej, o znacznie ograniczonej przepustowości i ogromnych latencjach. Każdy kolejny węzeł niezbędny do połączenia użytkownika z serwerem końcowym może dokładać cennych milisekund czasu odpowiedzi. 

Autorzy IPFS zauważają w swojej pracy (\cite{ipfsWP}) także inne słabości obecnego internetu – sieć oparta o HTTP jest w prawdzie zdecentralizowana, jako iż treści rozdzielane są pomiędzy miliony węzłów, od gigantycznej sieci Amazon Web Services aż po mikroserwery stojące w domach pasjonatów; brakuje jej jednak faktycznego rozproszenia: ta infrastruktura nie jest gotowa „by design” na przyjmowanie gigantycznego ruchu, nie jest w stanie efektywnie przechowywać i udostępniać wielkich zestawów danych; jest również podatna na „znikanie” danych, jako iż awaria pojedynczego dysku twardego może zatrzymać udostępnianie całej witryny. 

% TODO - Tu trzeba dopisać jakieś przełamanie w kolejną sekcję

\section{Zarys teoretyczny}
\label{sec:teoriaRozproszenia}


\section{Historia i istniejące narzędzia}
\label{sec:narzedzia}

\section{Podstawa projektu}
\label{sec:podstawaProjektu}
