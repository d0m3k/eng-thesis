\chapter{Podsumowanie}
\label{cha:summary}

Celem tej pracy było opracowanie rozwiązania rozpraszającego statyczne dane stron internetowych w celu zwiększenia wydajności przeglądania. System musiał wykazać się wysoką wydajnością i prostotą użytkowania -- jest w końcu próbą uczynienia IPFS przystępnym dla użytkownika końcowego, zastosowaniem leżącej pod nim technologii w celu popularyzacji rozwiązań zdecentralizowanych.

% TODO - upewnij się co do tej wydajności x)
Testy wykazują, że w pewnych okolicznościach zastosowanie dHTTP ma zauważalny, pozytywny wpływ na wydajność. Dzięki swojej konstrukcji, projekt w sposób przezroczysty dla użytkownika optymalizuje sposób w jaki przegląda on często odwiedzane strony, a zdecentralizowana architektura obniża obciążenie serwerów w miarę rozrastania się sieci węzłów.

\section{Potencjalne kierunki rozwoju}
\label{sec:future}

Zaprezentowany projekt nie wyczerpuje możliwości rozwoju.

Choć wstępne eksperymenty okazały się mieć satysfakcjonujące wyniki, dało się zauważyć negatywny wpływ węzłów, które mają niższą przepustowość niż klienci z nich korzystający. Trudno też ocenić sprawność systemu wielkiej skali na podstawie tak małej próby. W celu faktycznej walidacji utworzonych rozwiązań, konieczne będą ekstensywne testy, które mogą doprowadzić do istotnych wniosków. Wydajność sieci mogłaby również zostać znacznie poprawiona dzięki rozwojowi algorytmów zaproponowanych w sekcji \ref{sec:dataPropagation}, zapewniających optymalniejszy rozkład danych w sieci. Ponadto, w celu zapewnienia płynności działania systemu, metadane klastra powinny zawierać w sobie informacje o {\em reputacji} poszczególnych węzłów. Podczas gdy węzły serwerowe z reguły zostają uruchomione i będą działać, mając szczyty obciążenia czy wymagań zależne głównie od obciążenia całej sieci (lub ewentualnego alternatywnego oprogramowania serwerowego na nich uruchamianego), węzły klienckie podlegać będą prawdopodobnie o wiele większej dynamice, z racji licznych odczytów dużych plików i podejścia użytkownika, które może zakładać częste otwieranie i zamykanie przeglądarki. Węzły serwerowe i klienci powinny mieć zatem różne wartości bazowe przy ocenie ich sprawności w sieci -- reputacja oprogramowania klienckiego powinna być domyślnie niższa, i szybciej reagować na zmiany jej wartości.

Kluczowe dla trafienia do szerokiego grona odbiorców może być pokonanie ograniczeń narzucanych przez API Google Chrome. W obecnym kształcie wymusza ono instalację osobnego oprogramowania w celu uruchomienia węzła sieci, co jest problematyczne dla większości użytkowników internetu.