\chapter{Wprowadzenie}
\label{cha:wprowadzenie}

Dzisiejsza informatyka jak nigdy stoi przed wyzwaniami związanymi z wydajnością, wymuszanymi poprzez miliony klientów z szerokopasmowym dostępem do sieci. Ich rozwiązywanie nie jest już możliwe przez wzmacnianie podzespołów pojedynczych komputerów -- wymaga odpowiedniego sposobu projektowania, który skupia się na możliwościach maksymalnego rozłożenia i łatwego skalowania obciążenia. Ta praca proponuje rozwiązanie, które wpisuje się w ten motyw i w przystępny klientom sposób może znacznie odciążyć rozwijające się części sieci web.

\section{Cele pracy}
\label{sec:celePracy}

% TODO - tę sekcję prawdopodobnie trzeba przepisać, uwzględniając efekty pracy w dalszym ciągu tekstu + fakt że bardzo bardzo rozpisuję sie nad tym co zastępujemy

Celem tej pracy jest stworzenie protokołu i oprogramowania pozwalającego na rozproszony i bezpieczny dostęp do współdzielonych zasobów, w warstwie użytkowej odpowiadającego obecnemu Hypertext Transfer Protocol (HTTP). W toku pracy, system określany będzie skrótem \textbf{dHTTP} -- distributed HTTP.

% prawdopodobnie trzeba to wynieść do któregoś z kolejnych rozdziałów, bo już jest zbyt granuralne

Głównym założeniem projektu jest rozłożenie obciążenia stron, które nagle zyskują popularność -- często konieczna dla właścicieli takich stron jest inwestycja w szybsze łącza czy więcej sprzętu, wiążąca się ze sporymi kosztami. Kosztami które mogą zresztą nie znaleźć uzasadnienia -- dodatkowy sprzęt pomoże w chwilach szczytowego ruchu, będzie jednak przez większość czasu leżeć odłogiem, wciąż generując koszty łącz i prądu.

Częściowym rozwiązaniem tego problemu są chmury i udostępniany przez nie {\em autoscaling} -- użytkownicy, zamiast utrzymywać własną infrastrukturę, mogą nie tylko zdać się na serwery zarządzane przez zewnętrzny podmiot, ale także wykorzystać mechanizmy automatycznego rozszerzania i zmniejszania ilości urządzeń czy wykorzystywanego łącza, w zależności od obciążenia całego klastra. To pozwala na drastyczną minimalizację kosztów i jest rozwiązaniem coraz chętniej stosowanym także przez podmioty o gigantycznym ruchu sieciowym, jak serwis streamingowy Netflix (\cite{AWSAs}).

Wciąż jednak są to rozwiązania obciążające właściciela serwera, co może stanowić problem w przypadku projektów hobbystycznych czy krajów rozwijających się -- koszt autoscalingu w czasach szczytu może okazać się nieproporcjonalnie wysoki w stosunku do czasów spokojnego ruchu.

% poprzedni komentarz kończy się TU

System zaprezentowany w poniższej pracy -- dHTTP -- ma posłużyć jako samoskalująca się, rozproszona alternatywa do tego podejścia. Musi wziąć on pod uwagę optymalne rozłożenie danych, zabezpieczenie przed zmienianiem zawartości zasobów przez niepowołane podmioty (hashe i podpisy kryptograficzne), szybki i możliwie najmniej scentralizowany sposób współdzielenia informacji o stanie systemu ({\em kto jest online i może dać mi plik \texttt{logo.png} z hosta \texttt{example.com}?}) i aktualizacjach zawartości oraz stronę kliencką, będącą w tym przypadku wtyczką do przeglądarki Google Chrome, nakładającą warstwę dHTTP na polecenia pobierania zasobów. Z punktu widzenia użytkownika system jest {\em przezroczysty}: zostanie  węzłem i klientem sieci dHTTP polega wyłącznie na instalacji wtyczki, działającej autonomicznie w tle. Oprócz aplikacji klienckiej, dHTTP udostępniony jest także w trybie {\em headless} -- węzeł może zostać częścią klastra w sposób automatyczny, bez użycia przeglądarki.

\section{Zawartość}
\label{sec:zawartosc}

Rozdział 1. stanowi krótkie wprowadzenie i definicję celu pracy. 

Rozdział 2. zawiera rozważania na temat historycznych i teoretycznych podstaw systemów rozproszonych, oraz przykłady najpopularniejszych narzędzi je implementujących. Wprowadza także do konceptualnych i narzędziowych podstaw implementacji systemu dHTTP. 

Rozdział 3. jest szczegółowym opisem wymagań, projektu, modułów, a także implementacji samego systemu, w szczegółach omawiającym funkcjonalności, biblioteki i narzędzia pozwalające na działanie aplikacji, napotkane w toku projektowania problemy i ich rozwiązania, rys architektoniczny sposobu przechowywania danych czy ich propagacji, z uwzględnieniem implementacji niezależnego węzła ({\em headless mode}) i rozwiązania dla klienta końcowego.

Rozdział 4. to próba walidacji systemu dHTTP – analiza użytkowości systemu, zbiór badań nad wydajnością (z uwzględnieniem różnych mechanizmów propagacji danych) i propozycji na jej dalsze optymalizacje.

Rozdział 5. zawiera krótkie podsumowanie, podkreślające osiągnięte cele i potencjalne kierunki dalszego rozwoju projektu.

