\chapter{Testy wydajności}
\label{cha:testyIOptymalizacje}

W ramach testu systemu skonstruowano stronę testową, która składa się z pięciu wysokiej jakości zdjęć. Łączny rozmiar strony internetowej i zdjęć wynosi 39.1 MB.

\begin{figure}[h]
    \centering
    \includegraphics[scale=0.3]{perf-page.png}

    \caption{Zrzut ekranu prezentujący stronę testową aplikacji.}
    \label{fig:test-page}
\end{figure}

W celu oceny wydajności systemu, uruchomiony zostanie serwer \texttt{http} w środowisku node.js, na instancji EC2 znajdującej się w obszarze dostępności US East (Ohio) Amazon Web Services. Przy wykorzystaniu pakietu \texttt{throttle} (\cite{npmThrottle}) możliwe jest symulowanie ograniczonej przepustowości zarówno serwera końcowego, jak i pojedynczych węzłów sieci dHTTP.

\section{Rezultaty testów}
\begin{table}[H]
\centering
    \begin{tabular}{|c|c|c|c|c|c|}
        \hline
         & & \multicolumn{2}{c|}{\textbf{Serwer}} \\ \hline
         & & \textbf{Brak ograniczeń} & \textbf{1 Mbps} \\ \hline
         \multicolumn{2}{|c|}{\textbf{Brak dHTTP}} & 15.73s & 328.1s \\ \hline
         \multicolumn{2}{|c|}{\textbf{Węzeł lokalny, 0 sąsiadów}} & 15.52s & 335.2s \\ \hline \hline
         \textbf{1 sąsiad} & \textbf{Brak ograniczeń} & 16.31s & 18.42s \\ \hline
         \multirow{2}{*}{\textbf{5 sąsiadów}} & \textbf{Brak ograniczeń} & 14.45s & 15.32s \\ 
          & \textbf{1 Mbps} & 73.12s & 74.31s \\  \hline
    \end{tabular}
    \label{tab:test-results}
    \caption{Wyniki są uśrednieniem pięciu prób wczytania plików bez użycia pamięci podręcznej przeglądarki.}
\end{table}

\section{Wnioski}

Przy pełnej przepustowości sieci efekt działania dHTTP jest niemal niezauważalny (co wynika z nikłego obciążenia węzłów posiadających upload na średnim poziomie 88 Mbps według SpeedTest.net); warto jednak zwrócić uwagę na znaczną poprawę przepustowości przy ograniczonej szybkości oryginalnego serwera -- w tym przypadku, zastosowanie dHTTP zaowocowało przyrostem szybkości na poziomie 442\% względem pobierania bez wsparcia pięciu węzłów sąsiedzkich.

Warto zwrócić również uwagę na perspektywę optymalizacji systemu -- w przypadku braku ograniczeń łącza serwera, ale pięciu węzłów o przepustowości 1Mbps, szybkość transferu jest znacznie niższa, niż gdyby użytkownik użył oryginalnego serwera. Uwzględnienie metryki średniej szybkości łącza węzłów dHTTP mogłoby pozytywnie wpływać na klientów wyposażonych w szerokopasmowe połączenie. 