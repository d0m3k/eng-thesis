\chapter{Testy wydajności}
\label{cha:testyIOptymalizacje}

W ramach testu systemu skonstruowano stronę testową, która składa się z pięciu wysokiej jakości zdjęć. Łączny rozmiar strony internetowej i zdjęć wynosi 41 MB.

\begin{figure}[h]
    \centering
    \includegraphics[scale=0.3]{perf-page.png}

    \caption{Zrzut ekranu prezentujący stronę testową aplikacji.}
    \label{fig:test-page}
\end{figure}

W celu oceny wydajności systemu, uruchomiony zostanie serwer \texttt{http} w środowisku node.js, na serwerze znajdującym się w obszarze dostępności US East (Ohio) Amazon Web Services. Przy wykorzystaniu pakietu \texttt{throttle} (\cite{npmThrottle}) możliwe będzie symulowanie ograniczonej przepustowości zarówno serwera końcowego, jak i pojedynczych węzłów sieci dHTTP. Wyniki są uśrednieniem pięciu prób wczytania plików, bez użycia pamięci podręcznej przeglądarki.

\section{Rezultaty testów}
\begin{table}[H]
\centering
    \begin{tabular}{|c|c|c|c|c|c|}
        \hline
         & & \multicolumn{2}{c|}{\textbf{Serwer}} \\ \hline
         & & \textbf{Brak ograniczeń} & \textbf{1 Mbps} \\ \hline
         \multicolumn{2}{|c|}{\textbf{Brak dHTTP}} & 15.73s & cell2 \\ \hline
         \multicolumn{2}{|c|}{\textbf{Węzeł lokalny, 0 sąsiadów}} & 15.52s & cell2 \\ \hline \hline
         \multirow{2}{*}{\textbf{1 sąsiad}} & \textbf{Brak ograniczeń} & 17.31s & cell5 \\ 
          & \textbf{1 Mbps} & cell9 & cell5 \\  \hline
         \multirow{2}{*}{\textbf{5 sąsiadów}} & \textbf{Brak ograniczeń} & 12.45s & cell5 \\ 
          & \textbf{1 Mbps} & cell9 & cell5 \\  \hline
    \end{tabular}
    \label{tab:test-results}
\end{table}

\section{Wnioski}